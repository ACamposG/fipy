%% -*-TeX-*-
 % ###################################################################
 %  FiPy - a finite volume PDE solver in Python
 % 
 %  FILE: "howtoread.tex"
 %                                    created: 8/13/05 {10:34:56 AM} 
 %                                last update: 8/18/05 {4:12:28 PM} 
 %  Author: Jonathan Guyer
 %  E-mail: guyer@nist.gov
 %    mail: NIST
 %     www: http://www.ctcms.nist.gov/fipy/
 %  
 % ========================================================================
 % This document was prepared at the National Institute of Standards
 % and Technology by employees of the Federal Government in the course
 % of their official duties.  Pursuant to title 17 Section 105 of the
 % United States Code this document is not subject to copyright
 % protection and is in the public domain.  howtoread.tex
 % is an experimental work.  NIST assumes no responsibility whatsoever
 % for its use by other parties, and makes no guarantees, expressed
 % or implied, about its quality, reliability, or any other characteristic.
 % We would appreciate acknowledgement if the document is used.
 % 
 % This document can be redistributed and/or modified freely
 % provided that any derivative works bear some notice that they are
 % derived from it, and any modified versions bear some notice that
 % they have been modified.
 % ========================================================================
 % See the file "license.terms" for information on usage and 
 % redistribution of this file, and for a DISCLAIMER OF ALL WARRANTIES.
 %  
 %  Description: 
 % 
 %  History
 % 
 %  modified   by  rev reason
 %  ---------- --- --- -----------
 %  2005-08-13 JEG 1.0 original
 % ###################################################################
 %%

\chapter{How To Read This Manual}
\label{chp-howtoread}

This chapter will illustrate the typesetting conventions used
throughout this manual.

\begin{FiPyDocumentationExample}
%     \renewcommand{\section}{\section*}
    
%     \chapter*{Package \EpydocDottedName{fipy.package}}
    \chapterheadstart
    {\printchaptertitle{Package \EpydocDottedName{fipy.package}}
     \afterchaptertitle
    }
    
    Each chapter describes one of the main sub-packages of the
    \EpydocDottedName{fipy} package.  The
    \EpydocDottedName{fipy.package} package can be found in the
    directory \path{fipy/package/}.  In a few cases, there will be
    packages within packages, \emph{e.g.}
    \EpydocDottedName{fipy.package.subpackage} located in
    \path{fipy/package/subpackage/}.  These sub-packages
    will not be given their own chapters; rather, their contents will
    be described in the chapter for their containing package.
    
    \input tutorial/latex/fipy.package.object-module
    
%     \section*{Module \EpydocDottedName{fipy.package.object}}
% 
%     This module can be found in the file \path{fipy/package/object.py}.
% 
%     %%%%%%%%%%%%%%%%%%%%%%%%%%%%%%%%%%%%%%%%%%%%%%%%%%%%%%%%%%%%%%%%%%%%%%%%%%%
%     %%                           Class Description                           %%
%     %%%%%%%%%%%%%%%%%%%%%%%%%%%%%%%%%%%%%%%%%%%%%%%%%%%%%%%%%%%%%%%%%%%%%%%%%%%
% 
%     \subsection*{Class \EpydocDottedName{Object}}
%     
%     With very few exceptions, the name of a class will be the
%     capitalized form of the module it resides in.  
%     
%     
%     \Python{} is an object-oriented language and the \FiPy{} framework
%     is composed of objects or classes.  Knowledge of object-oriented
%     programming (OOP) is not necessary to use either \Python{} or
%     \FiPy{}, but a few concepts are useful.  OOP involves two main
%     ideas:
%     \begin{description}
%         \item[encapsulation] an object binds data with actions or 
%         ``methods''. In most cases, you will not work with an 
%         object's data directly; instead, you will set, retrieve, or 
%         manipulate the data using the object's methods.
%     
%         \item[inheritance] objects are derived or inherited from more 
%         abstract objects. Common behaviors or data are defined in 
%         base objects and specific behaviors or data are either added 
%         or modified in derived objects.
%     \end{description}
%     
%     \begin{tabular}{cccccc}
%     \multicolumn{2}{r}{\settowidth{\BCL}{\EpydocDottedName{fipy.package.base.Base}}\multirow{2}{\BCL}{\EpydocHyperlink{}{\EpydocDottedName{fipy.package.base.Base}}}}
%     &&
%       \\\cline{3-3}
%       &&\multicolumn{1}{c|}{}
%     &&
%       \\
%     &&\multicolumn{2}{l}{\textbf{\EpydocDottedName{Object}}}
%     \end{tabular}
% 
%     \begin{EpydocDescriptionShortList}{Known Subclasses}%
%         \EpydocHyperlink{}{\EpydocDottedName{ModuleSubClass}}%
%         \and \EpydocHyperlink{}{\EpydocDottedName{ModuleSubSubClass}}%
%         \and \EpydocDottedName{_ModulePrivateSubClass}%
%     \end{EpydocDescriptionShortList}
% 
%     \vdots
% 
%     %%%%%%%%%%%%%%%%%%%%%%%%%%%%%%%%%%%%%%%%%%%%%%%%%%%%%%%%%%%%%%%%%%%%%%%%%%%
%     %%                                Methods                                %%
%     %%%%%%%%%%%%%%%%%%%%%%%%%%%%%%%%%%%%%%%%%%%%%%%%%%%%%%%%%%%%%%%%%%%%%%%%%%%
% 
%       \subsubsection*{Methods}
%       
%       Methods are functions that are attached to objects and that have
%       direct access to the data of those objects.  Rather than 
%       passing the object data as an argument to a function:
%       \begin{quote}
%       \begin{verbatim}
% >>> fn(object.data, arg1, arg2, ...)\end{verbatim}
%       \end{quote}
%       you instruct an object to invoke an appropriate method
%       \begin{quote}
%       \begin{verbatim}
% >>> object.meth(arg1, arg2, ...)\end{verbatim}
%       \end{quote}
% 
%       Methods whose names begin and end with ``\verb+__+'' are
%       special.  You won't ever need to call these methods directly,
%       but \Python{} will invoke them for you under certain
%       circumstances.
% 
%     \begin{EpydocFunctionGroup}[]
%     \begin{EpydocFunction}{\EpydocDottedName{__init__}}{\Param{self}%
%         \and \Param{arg1}%
%         \and \Param[None]{arg2}%
%         \and \Param["string"]{arg3}%
%         \and \ldots}%
%         \begin{EpydocDocstring}
%             This method is invoked when you create an object, as in
%             
%             \begin{quote}
%             \begin{verbatim}
% >>> obj = Object(arg1 = something, arg3 = somethingElse, ...)\end{verbatim}
%             \end{quote}
%             
%             \begin{EpydocFunctionParameters}{xxxxx}
%                   \item[self] this special argument
%                   refers to the object that is being created and is
%                   supplied automatically by the \Python{} interpreter
%                   to all methods.  You don't ever need to (and should
%                   not) specify it yourself.
%                   \item[arg1] this argument is 
%                   required, and must either be listed first 
%                   \begin{quote}
%                   \begin{verbatim}
% >>> obj = Object(val1, val2, ...)\end{verbatim}
%                   \end{quote}
%                   or be listed by name
%                   \begin{quote}
%                   \begin{verbatim}
% >>> obj = Object(arg2 = val2, arg1 = val1, ...)\end{verbatim}
%                   \end{quote}
% 
%                   \item[arg2] this argument may be 
%                   omitted, in which case it will be assigned a 
%                   default value of \verb+None+.
% 
%                   \item[arg3] this argument may be 
%                   omitted, in which case it will be assigned a 
%                   default value of \verb+"string"+.
%             \end{EpydocFunctionParameters}
%         \end{EpydocDocstring}
%     \end{EpydocFunction}
% 
%     \begin{EpydocFunction}{\EpydocDottedName{__add__}}{\Param{self}%
%         \and \Param{other}}%
%         \begin{EpydocDocstring}
%             This method is invoked when you add \verb+other+ to an 
%             \verb+Object+:
%             \begin{quote}
%             \begin{verbatim}
% >>> a = Object(...) + other\end{verbatim}
%             \end{quote}
%         \end{EpydocDocstring}
%     \end{EpydocFunction}
% 
%     \begin{EpydocFunction}{\EpydocDottedName{__radd__}}{\Param{self}%
%         \and \Param{other}}%
%         \begin{EpydocDocstring}
%             This method is invoked when you add an 
%             \verb+Object+ to \verb+other+:
%             \begin{quote}
%             \begin{verbatim}
% >>> a = other + Object(...)\end{verbatim}
%             \end{quote}
%         \end{EpydocDocstring}
%     \end{EpydocFunction}
% 
% \end{EpydocFunctionGroup}
\end{FiPyDocumentationExample}
