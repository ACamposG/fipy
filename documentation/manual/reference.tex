%% -*-TeX-*-
 % ###################################################################
 %  FiPy - a finite volume PDE solver in Python
 % 
 %  FILE: "reference.tex"
 %                                    created: 9/29/04 {11:03:55 AM} 
 %                                last update: 9/30/04 {3:52:34 PM} 
 %  Author: Jonathan Guyer
 %  E-mail: guyer@nist.gov
 %  Author: Daniel Wheeler
 %  E-mail: daniel.wheeler@nist.gov
 %    mail: NIST
 %     www: http://ctcms.nist.gov
 %  
 % ========================================================================
 % This document was prepared at the National Institute of Standards
 % and Technology by employees of the Federal Government in the course
 % of their official duties.  Pursuant to title 17 Section 105 of the
 % United States Code this document is not subject to copyright
 % protection and is in the public domain.  reference.tex
 % is an experimental work.  NIST assumes no responsibility whatsoever
 % for its use by other parties, and makes no guarantees, expressed
 % or implied, about its quality, reliability, or any other characteristic.
 % We would appreciate acknowledgement if the document is used.
 % 
 % This document can be redistributed and/or modified freely
 % provided that any derivative works bear some notice that they are
 % derived from it, and any modified versions bear some notice that
 % they have been modified.
 % ========================================================================
 % See the file "license.terms" for information on usage and 
 % redistribution of this file, and for a DISCLAIMER OF ALL WARRANTIES.
 %  
 %  Description: 
 % 
 %  History
 % 
 %  modified   by  rev reason
 %  ---------- --- --- -----------
 %  2004-09-29 JEG 1.0 original
 % ###################################################################
 %%

\documentclass[letterpaper]{book}

\usepackage{fipy}

\begin{document}

% \doparttoc

% \crop[frame,axes]

\frontmatter

% \layout

% \settypeblocksize{9in}{7in}{*}
% \setlrmargins{*}{*}{*}
% \setulmargins{*}{*}{*}

% \settrimmedsize{8.5in}{11in}{*}       % pi/2
% % \settypeblocksize{40\onelineskip}{*}{0.61803}  % golden ratio
% \settypeblocksize{33\onelineskip}{*}{0.61803}  % golden ratio
% \setlength{\trimtop}{0pt}
% \setlength{\trimedge}{\stockwidth}
% \addtolength{\trimedge}{-\paperwidth}
% \addtolength{\trimedge}{-1.5in}
% \setlrmargins{*}{*}{2}
% % \setlrmargins{*}{1in}{*}
% \setulmargins{5\onelineskip}{*}{*}
% \setheadfoot{3\onelineskip}{3\onelineskip}
% \setheaderspaces{\onelineskip}{*}{*}
% \setmarginnotes{17pt}{65pt}{\onelineskip}

% \checkandfixthelayout

% \fixpdflayout

% \setlength{\parindent}{0ex}

\title{\FiPy{}}
\subtitle{Programmer's Reference}

\author{Daniel Wheeler \and Jonathan E. Guyer \and James A. Warren}
\affiliation{Metallurgy Division 
and the Center for Theoretical and Computational Materials Science 
\and Materials Science and Engineering Laboratory}

\maketitle

\vspace*{\fill}

\input license

% \rule{\textwidth}{1pt}
% 
% \input disclaimer

\maxtocdepth{all} 
\tableofcontents
% \faketableofcontents

\mainmatter

%%%%%%%%%%%%%%%%
% I. INTRODUCTION
%%%%%%%%%%%%%%%%

% \include{examples}
% 
% \include{theory}
% 
% \include{code}
 
% \appendix
% 
% \part{APIs}
% 
% \settocdepth{chapter} 

% \parttoc

\input api

% \settocdepth{all} 

\backmatter

%%%%%%%%%%%%%%%%%%%%%%%%%%%%%%%%%%%%%%%%%%%%%%%%%%%%%%%%%%%%%%%%%%%%%%%%%%%
%%                                 Index                                 %%
%%%%%%%%%%%%%%%%%%%%%%%%%%%%%%%%%%%%%%%%%%%%%%%%%%%%%%%%%%%%%%%%%%%%%%%%%%%

% \pdfbookmark[-1]{Index}{index}
\printindex


\end{document}
