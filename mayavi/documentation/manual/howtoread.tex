%% -*-TeX-*-
 % ###################################################################
 %  FiPy - a finite volume PDE solver in Python
 % 
 %  FILE: "howtoread.tex"
 %                                    created: 8/13/05 {10:34:56 AM} 
 %                                last update: 10/9/08 {2:36:43 PM} 
 %  Author: Jonathan Guyer
 %  E-mail: guyer@nist.gov
 %    mail: NIST
 %     www: http://www.ctcms.nist.gov/fipy/
 %  
 % ========================================================================
 % This document was prepared at the National Institute of Standards
 % and Technology by employees of the Federal Government in the course
 % of their official duties.  Pursuant to title 17 Section 105 of the
 % United States Code this document is not subject to copyright
 % protection and is in the public domain.  howtoread.tex
 % is an experimental work.  NIST assumes no responsibility whatsoever
 % for its use by other parties, and makes no guarantees, expressed
 % or implied, about its quality, reliability, or any other characteristic.
 % We would appreciate acknowledgement if the document is used.
 % 
 % This document can be redistributed and/or modified freely
 % provided that any derivative works bear some notice that they are
 % derived from it, and any modified versions bear some notice that
 % they have been modified.
 % ========================================================================
 % See the file "license.terms" for information on usage and 
 % redistribution of this file, and for a DISCLAIMER OF ALL WARRANTIES.
 %  
 %  Description: 
 % 
 %  History
 % 
 %  modified   by  rev reason
 %  ---------- --- --- -----------
 %  2005-08-13 JEG 1.0 original
 % ###################################################################
 %%

\chapter{How To Read This Manual}
\label{chp-howtoread}

This chapter will illustrate the conventions used
throughout this manual.

\begin{FiPyDocumentationExample}
%     \renewcommand{\section}{\section*}
    
%     \chapter*{Package \EpydocDottedName{fipy.package}}
    \chapterheadstart
    {\printchaptertitle{Package \EpydocDottedName{fipy.package}}
     \afterchaptertitle
    }
    
    Each chapter describes one of the main sub-packages of the
    \EpydocDottedName{fipy} package.  The sub-package
    \EpydocDottedName{fipy.package} can be found in the
    directory \path{fipy/package/}.  In a few cases, there will be
    packages within packages, \emph{e.g.}
    \EpydocDottedName{fipy.package.subpackage} located in
    \path{fipy/package/subpackage/}.  These sub-sub-packages
    will not be given their own chapters; rather, their contents will
    be described in the chapter for their containing package.

    \import{tutorial/latex/}{fipy.package.base-module}
%     \input tutorial/latex/fipy.package.base-module
\end{FiPyDocumentationExample}

\begin{FiPyDocumentationExample}
    \import{tutorial/latex/}{fipy.package.object-module}
%     \input tutorial/latex/fipy.package.object-module
\end{FiPyDocumentationExample}

% \begin{FiPyDocumentationExample}
%     \input tutorial/latex/fipy.package.objectDescendant-module
% \end{FiPyDocumentationExample}

