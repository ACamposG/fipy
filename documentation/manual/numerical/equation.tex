\section{General Conservation Equation}

A general conservation equation solved using \FiPy{} can include any
combination of the following terms,
\begin{equation}
  \underbrace{
    \frac{\partial (\rho \phi)}{\partial t}
  }_{\text{transient}}
=
\underbrace{
  \nabla \cdot \left( \vec{u} \phi \right)
}_{\text{convection}}
+
\underbrace{
  \nabla \cdot \left( \Gamma \nabla \phi \right) 
}_{\text{diffusion}}
+
\underbrace{
  \nabla \cdot \left( \Gamma_2 \nabla 
    \left( \nabla \cdot \left( \Gamma_1 \nabla \phi \right) \right) \right)
}_{\text{4th order}}
+
\underbrace{
  L_n \phi
}_{\text{nth order}}
+
\underbrace{
  S_{\phi}
}_{source}
\label{eqn:num:gen}
\end{equation}
where $\rho$, $\vec{u}$, $\Gamma_i$ represent coefficients in their
respective terms. These coefficients can be arbitrary functions of any
parameters or variables in the system. The variable $\phi$ represents
the quantity that is being solved for by the equation. The operator
$L_n$ is arbitrary where $n=2$ represents the diffusion term and $n=4$
represents the fourth order term. Higher order diffusion type terms
are possible using this operator.



