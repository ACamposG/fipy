%% -*-TeX-*-
 % ###################################################################
 %  FiPy - a finite volume PDE solver in Python
 % 
 %  FILE: "fipy.tex"
 %                                    created: 4/1/04 {2:58:37 PM} 
 %                                last update: 7/3/07 {5:01:19 PM} 
 %  Author: Jonathan Guyer
 %  E-mail: guyer@nist.gov
 %  Author: Daniel Wheeler
 %  E-mail: daniel.wheeler@nist.gov
 %    mail: NIST
 %     www: http://ctcms.nist.gov
 %  
 % ========================================================================
 % This document was prepared at the National Institute of Standards
 % and Technology by employees of the Federal Government in the course
 % of their official duties.  Pursuant to title 17 Section 105 of the
 % United States Code this document is not subject to copyright
 % protection and is in the public domain.  fipy.tex
 % is an experimental work.  NIST assumes no responsibility whatsoever
 % for its use by other parties, and makes no guarantees, expressed
 % or implied, about its quality, reliability, or any other characteristic.
 % We would appreciate acknowledgement if the document is used.
 % 
 % This document can be redistributed and/or modified freely
 % provided that any derivative works bear some notice that they are
 % derived from it, and any modified versions bear some notice that
 % they have been modified.
 % ========================================================================
 % See the file "license.terms" for information on usage and  redistribution of
 % this file, and for a DISCLAIMER OF ALL WARRANTIES.
 %  
 % ###################################################################
 %%

% \documentclass[tocAsPDFpart]{fipy}
\documentclass[letterpaper,twoside,openright,10pt]{memoir}

% \documentclass[letterpaper]{book}

\usepackage[tocAsPDFchapter]{fipy}


\usepackage{xr-hyper}
\externaldocument[API-]{reference}

% \usepackage{mdwlist}
% \newenvironment{faq}{%
%     \begin{basedescript}{%
%         \setlength\labelwidth\textwidth%
%         \desclabelstyle{\multilinelabel}%
%         \renewcommand{\makelabel}[1]{\bfseries##1\newline}%
%     }%
% }{%
%     \end{basedescript}%
% }

\usepackage{moreverb}

\begin{document}

\dominitoc
\doparttoc

\frontmatter

\title{\FiPy{}}
%\subtitle{A Finite Volume PDE Solver Using Python}
\subtitle{User's Guide}

% \author{Daniel Wheeler \and Jonathan E. Guyer \and James A. Warren}
% \affiliation{Metallurgy Division \and
% and the Center for Theoretical and Computational Materials Science 
% \and Materials Science and Engineering Laboratory}

\author{Daniel Wheeler \andnext Jonathan E. Guyer \andnext James A. Warren}
\affiliation{Metallurgy Division \and
and the Center for Theoretical and Computational Materials Science 
\and Materials Science and Engineering Laboratory}

% \begin{titlingpage}
\maketitle
% \thispagestyle{empty}
\thispagestyle{titlingpage}

\newpage

\vspace*{\fill}
\input license

\rule{\textwidth}{0.1pt}

\thispagestyle{empty}
\input disclaimer
% \end{titlingpage}

\maxtocdepth{all} 
\tableofcontents
% \faketableofcontents

\mainmatter

%%%%%%%%%%%%%%%%
% I. INTRODUCTION
%%%%%%%%%%%%%%%%

\part{Introduction}

\settocdepth{chapter} 

\renewcommand{\ptctitle}{Introduction Contents}

\parttoc

% Technical Overview

\chapter{Overview}

\inputencoding{latin1}

\input readme  

\section{Mailing List}

\input mail

% Installation

\chapter{Installation and Usage}
\label{chap:Installation}

\inputencoding{latin1}

\input installation

\newpage

\section{Mac OS X Installation}
\label{sec:MacOSXInstallation}

\inputencoding{latin1}

\input macosx-installation

\newpage

\section{Windows Installation}
\label{sec:WindowsInstallation}

\inputencoding{latin1}

\input windows-installation

\section{Subversion tags}
\label{sec:SVN}

\input svn

% Numerical

\chapter{Theoretical and Numerical Background}
\label{chap:Numerics}

\inputencoding{ascii}

\input numerical/numerical

% Implementation

\chapter{Design and Implementation}
\label{chap:Design}

\inputencoding{latin1}

\input introduction

% Efficiency

\chapter{Efficiency}
\label{chap:Efficiency}

\inputencoding{latin1}

\input efficiency

\chapter{Frequently Asked Questions}
\label{chap:FAQ}

\input faq

% TODO List

% \chapter{Future Work}
% \label{chap:ToDo}
% 
% \inputencoding{latin1}
% 
% \input todolist

%%%%%%%%%%%%%%%%%
% 2. EXAMPLES
%%%%%%%%%%%%%%%%%

\settocdepth{all} 

\sloppy

\part{Examples}
\label{part:Examples}

% \chapter{Usage}
\input{examples/readme}

\settocdepth{chapter} 
\renewcommand{\ptctitle}{Example Contents}
\parttoc

\chapter{Diffusion Examples}

\input{examples/latex/examples.diffusion.mesh1D-module}
\label{sec:Examples:diffusion:mesh1D}
% \newpage
\input{examples/latex/examples.diffusion.mesh20x20-module}
% \newpage
\input{examples/latex/examples.diffusion.circle-module}
% \newpage
\input{examples/latex/examples.diffusion.electrostatics-module}
% \newpage
\input{examples/latex/examples.diffusion.nthOrder.input4thOrder1D-module}
% \newpage
\input{examples/latex/examples.diffusion.anisotropy-module}

\chapter{Convection Examples}

\input{examples/latex/examples.convection.exponential1D.mesh1D-module}
% \newpage
% \input{examples/latex/examples.convection.exponential1DBack.mesh1D-module}
% \newpage
\input{examples/latex/examples.convection.exponential1DSource.mesh1D-module}
% \newpage
% \input{examples/latex/examples.convection.exponential2D.mesh2D-module}
% \newpage
% \input{examples/latex/examples.convection.powerLaw1D.mesh1D-module}
% \newpage

\chapter{Phase Field Examples}

% \section*{Solidification Examples}
% \addstarredsection{Solidification Examples}
% \addcontentsline{toc}{section}{Solidification Examples} 

The phase field method is a ``diffuse interface'' technique for
modeling phase transformations and interface motion. Several good
review articles have been written on the subject
\cite{BoettingerReview:2002,ChenReview:2002,McFaddenReview:2002}.

\input{examples/latex/examples.phase.simple-module}
\input{examples/latex/examples.phase.binary-module}
\input{examples/latex/examples.phase.quaternary-module}

\input{examples/latex/examples.phase.anisotropy-module}
\label{sec:anisotropy}
% \newpage
\input{examples/latex/examples.phase.impingement.mesh40x1-module}
% \newpage
\input{examples/latex/examples.phase.impingement.mesh20x20-module}
\label{sec:impingement}
% \newpage
% \input{examples/latex/examples.phase.impingement.restart-module}
% \newpage
% \input{examples/latex/examples.phase.missOrientation.circle-module}
% \newpage
% \input{examples/latex/examples.phase.missOrientation.mesh1D-module}
% \newpage
% \input{examples/latex/examples.phase.missOrientation.modCircle-module}
% \newpage

% \section*{Electrochemistry Examples}
% \addstarredsection{Electrochemistry Examples} 
% \addcontentsline{toc}{section}{Electrochemistry Examples} 

% The following examples exhibit various parts of a model to study
electrochemical interfaces.  In a pair of papers, Guyer, Boettinger, Warren
and McFadden \cite{ElPhFI,ElPhFII} have shown that an electrochemical
interface can be modeled by an equation for the phase field \( \xi \)

\[
    \frac{\partial \xi}{\partial t}
    = 
    M_{\xi}\kappa_{\xi}\nabla^2 \xi
    - 
    M_{\xi}\sum_{j=1}^{n} C_j \left[
	p'(\xi) \Delta\mu_j^\circ
	+ g'(\xi) W_j
    \right]
    +
    M_{\xi}\frac{\epsilon'(\xi)}{2}\left(\nabla\phi\right)^2
\]

a set of diffusion equations for the concentrations \( C_j \), for \(
j = 2,\ldots, n-1 \), of the substitutional elements

\begin{align*}
    \frac{\partial C_j}{\partial t}
    &= D_{j}\nabla^2 C_j \\
    & \qquad + 
	D_{j}\nabla\cdot 
	\frac{C_j}{1 - \sum_{\substack{k=2\\ k \neq j}}^{n-1} C_k}
	\left\{
	    \sum_{\substack{i=2\\ i \neq j}}^{n-1} \nabla C_i
	    + 
	    C_n \left[
		p'(\xi) \Delta\mu_{jn}^{\circ}
		+ g'(\xi) W_{jn}
	    \right] \nabla\xi
	    +
	    C_n z_{jn} \nabla \phi
	\right\}
\end{align*}

a diffusion equation for the concentration \( C_{\text{e}^{-}} \) of
electrons

\[
    \frac{\partial C_{\text{e}^{-}}}{\partial t}
    = D_{\text{e}^{-}}\nabla^2 C_{\text{e}^{-}} \\
    + D_{\text{e}^{-}}\nabla\cdot 
	C_{\text{e}^{-}}
	\left\{
	    \left[
		p'(\xi) \Delta\mu_{\text{e}^{-}}^{\circ}
		+ g'(\xi) W_{\text{e}^{-}}
	    \right] \nabla\xi
	    +
	    z_{\text{e}^{-}} \nabla \phi
	\right\}
\]

and Poisson's equation for the electrostatic potential \( \phi \)

\[ 
    \nabla\cdot\left(\epsilon\nabla\phi\right) 
    +
    \rho
    = 0
\]

\( M_\xi \) is the phase field mobility, \( \kappa_\xi \) is the phase
field gradient energy coefficient, \( p'(\xi) =
30\xi^2\left(1-\xi\right)^2 \), and \( g'(\xi) =
2\xi\left(1-\xi\right)\left(1-2\xi\right) \).  For a given species \(
j \), \( \Delta\mu_j^{\circ} \) is the standard chemical potential
difference between the electrode and electrolyte for a pure material,
\( W_j \) is the magnitude of the energy barrier in the double-well
free energy function, \( z_j \) is the valence, and \( D_{j} \) is the
self diffusivity.  \( \Delta\mu_{jn}^{\circ} \), \( W_{jn} \), and \(
z_{jn} \) are the differences of the respective quantities \(
\Delta\mu_{j}^{\circ} \), \( W_{j} \), and \( z_{j} \) between
substitutional species \( j \) and the solvent species \( n \).  The
total charge is denoted by \( \rho \equiv \sum_{j=1}^n z_j C_j \).

The module \verb+fipy.models.elphf+ has been developed to solve this
coupled set of equations.  Although unresolved stiffnesses make the
full solution intractable in \FiPy{}, we can demonstrate the use of
various parts of the \verb+elphf+ module.


% \input{examples/latex/examples.elphf.phase-module}
% \newpage
% \input{examples/latex/examples.elphf.diffusion.mesh1D-module}
% \newpage
% \input{examples/latex/examples.elphf.diffusion.mesh1Ddimensional-module}
% \newpage
% \input{examples/latex/examples.elphf.diffusion.mesh2D-module}

% \newpage
% \input{examples/latex/examples.elphf.poisson-module}
% \newpage
% \input{examples/latex/examples.elphf.phaseDiffusion-module}
% \newpage


\chapter{Level Set Examples}

The Level Set Method (LSM) is a popular interface tracking
method. Further details of the LSM and descriptions of the algorithms
used in \FiPy{} can be found in Sethian's Level Set
book~\cite{levelSetBook}.

\input{examples/latex/examples.levelSet.distanceFunction.mesh1D-module}
% \newpage
% \input{examples/latex/examples.levelSet.distanceFunction.square-module}
% \newpage
\input{examples/latex/examples.levelSet.distanceFunction.circle-module}
% \newpage
% \input{examples/latex/examples.levelSet.distanceFunction.interior-module}
% \newpage
\input{examples/latex/examples.levelSet.advection.mesh1D-module}
% \newpage
\input{examples/latex/examples.levelSet.advection.circle-module}
% \newpage

\newpage
\section*{Superconformal Electrodeposition Examples}
\addcontentsline{toc}{section}{Superconformal Electrodeposition Examples} 

\input{examples/levelset/electrochem/readme}

%Electroplating is a deposition method widely used to fill high-aspect
%ratio features without seams or voids through the process of
%superconformal deposition, also called ``superfill.''  This process
%has been demonstrated to depend critically on the inclusion of
%additives in the electrolyte.  Recent publications propose ``Curvature
%Enhanced Accelerator Coverage'' (CEAC) as the mechanism behind the
%superfilling process~\cite{NIST:damascene:2001}.  In this mechanism,
%molecules that accelerate local metal deposition displace molecules
%that inhibit local metal deposition on the metal/electrolyte
%interface. For electrolytes that yield superconformal filling of fine
%features, this buildup happens relatively slowly because the
%concentration of accelerator species is much more dilute compared to
%the inhibitor species in the electrolyte.  The mechanism that leads to
%the increased rate of metal deposition along the bottom of the filling
%trench is the concurrent local increase of the accelerator coverage
%due to decreasing local surface area, which scales with the local
%curvature (hence the name of the mechanism).

\input{examples/latex/examples.levelSet.electroChem.simpleTrenchSystem-module}
\input{examples/latex/examples.levelSet.electroChem.gold-module}
\input{examples/latex/examples.levelSet.electroChem.leveler-module}
\input{examples/latex/examples.levelSet.electroChem.howToWriteAScript-module}

\chapter{Cahn-Hilliard Examples}

\input{examples/latex/examples.cahnHilliard.inputTanh1D-module}
% \newpage

\chapter{Fluid Flow Examples}

\input{examples/latex/examples.flow.stokesCavity-module}

\chapter{Converting from \FiPy{}~0.1 to \FiPy{}~1.0}
\label{chap:Update0.1to1.0}

\input{examples/latex/examples.update0_1to1_0-module}

\settocdepth{all} 

\backmatter

%%%%%%%%%%%%%%%%%%%%%%%%%%%%%%%%%%%%%%%%%%%%%%%%%%%%%%%%%%%%%%%%%%%%%%%%%%%
%%                                 Index                                 %%
%%%%%%%%%%%%%%%%%%%%%%%%%%%%%%%%%%%%%%%%%%%%%%%%%%%%%%%%%%%%%%%%%%%%%%%%%%%

\fussy

% \bibliographystyle{unsrt_no_title}
\bibliographystyle{fipy}
\bibliography{refs}

\sloppy
\raggedright
\printindex

\fussy

\appendix 

% Credits

\chapter*{Contributors}
% \addstarredpart{Contributors} 
\addcontentsline{toc}{part}{Contributors} 
\label{chap:Contributors}

\inputencoding{latin1}

\input credits

\end{document}
