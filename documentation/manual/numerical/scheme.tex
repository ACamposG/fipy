\subsection{Numerical Schemes}

The coefficients of equation~\eqref{eqn:num:dap} must remain positive since
an increase in a neighbouring value must result in an increase in $\phi_P$,
to obtain physically realistic solutions~\cite{patankar}. Also the coefficients of
equation~\eqref{eqn:num:dis} when written in the form of equation~\eqref{eqn:num:dap}
without using equation~\eqref{eqn:num:cdi} must remain positive.
Thus $a_{nb}>0$ and $a_{nb} + F_f>0$ must be satisfied.
If $P_f=F_f/D_f$
then to achieve physically realistic solutions the following
equality must be satisfied,
\begin{equation}
\frac{1}{1-\alpha_f} > P_f > -\frac{1}{\alpha_f}
\label{eqn:num:inq}
\end{equation}
Given equation~\eqref{eqn:num:inq} then $\alpha_f$ can be redefined depending
on which scheme is chosen. The value $P_f$ is the ratio between
convective strength and diffusive conductance. The various differencing schemes
are~\cite{croftphd}:
\begin{itemize}
%
%
%
\item The central differencing scheme,
\begin{equation}
\alpha_f = \frac{d_{Af}}{d_{Af}+d_{fP}}
\label{eqn:num:cds}
\end{equation}
In many circumstances with a structured mesh, $\alpha_f=1/2$ so that,
$|P_f|<2$ to satisfy equation~\eqref{eqn:num:inq}.
Thus the central differencing scheme is only
numerically stable for a low values of $P_f$.
%
%
%
\item The upwind Scheme,
\begin{eqnarray}
\alpha_f = 1 & \mbox{if} & P_f > 0 \\
\alpha_f = 0 & \mbox{if} & P_f < 0 
\label{eqn:num:ups}
\end{eqnarray}
Thus equation~\eqref{eqn:num:ups} satisfies the inequality in
equation~\eqref{eqn:num:inq} for all values of $P_f$.
However the solution over predicts the diffusive term leading to excessive
numerical smearing (False Diffusion).
%
%
%
\item The exponential scheme,
\begin{equation}
\alpha_f = \frac{(P_f-1)\exp{(P_f)}+1}{P_f(\exp{(P_f)}-1)}
\label{eqn:num:exs}
\end{equation}
This formulation can be derived from the exact solution~\cite{croftphd}
and thus guarantees
positive coefficients while not over predicting the diffusive terms.
However the computation of exponentials is slow and therefore a
faster scheme is usually used, especially in higher dimensions.
%
%
%
\item The hybrid scheme,
\begin{eqnarray}
\alpha_f = \frac{P_f-1}{P_f} & \mbox{if} & P_f > 2 \\
\alpha_f = \frac{1}{2} & \mbox{if} & |P_f| < 2 \\
\alpha_f = -\frac{1}{P_f} & \mbox{if} & P_f < -2
\label{eqn:num:hys}
\end{eqnarray}
The hybrid scheme formulated by allowing $P_f \rightarrow \infty$, $P_f \rightarrow 0$
and $P_f \rightarrow -\infty$ in the exponential scheme.
The hybrid scheme is an improvement on the upwind scheme, however it deviates
from the exponential scheme at $|P_f|=2$.
%
%
%
\item The power law scheme,
\begin{eqnarray}
\alpha_f = \frac{P_f-1}{P_f} & \mbox{if} & P_f > 10 \\
\alpha_f = \frac{(P_f-1)+(1-P/10)^5}{P_f} & \mbox{if} & 0 < P_f < 10 \\
\alpha_f = \frac{(1-P_f/10)^5 - 1}{P_f} & \mbox{if} & -10 < P_f < 0 \\
\alpha_f = -\frac{1}{P_f} & \mbox{if} & P_f < -10
\label{eqn:num:pls}
\end{eqnarray}
The power law scheme overcomes the inaccuracies of the hybrid scheme while
improving on the computational time for the exponential scheme. The power
law scheme is used in the work presented in this thesis.
\end{itemize}







