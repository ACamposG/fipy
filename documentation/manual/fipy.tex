%% -*-TeX-*-
 % ###################################################################
 %  FiPy - a finite volume PDE solver in Python
 % 
 %  FILE: "fipy.tex"
 %                                    created: 4/1/04 {2:58:37 PM} 
 %                                last update: 6/10/04 {9:57:13 PM} 
 %  Author: Jonathan Guyer
 %  E-mail: guyer@nist.gov
 %  Author: Daniel Wheeler
 %  E-mail: daniel.wheeler@nist.gov
 %    mail: NIST
 %     www: http://ctcms.nist.gov
 %  
 % ========================================================================
 % This document was prepared at the National Institute of Standards
 % and Technology by employees of the Federal Government in the course
 % of their official duties.  Pursuant to title 17 Section 105 of the
 % United States Code this document is not subject to copyright
 % protection and is in the public domain.  fipy.tex
 % is an experimental work.  NIST assumes no responsibility whatsoever
 % for its use by other parties, and makes no guarantees, expressed
 % or implied, about its quality, reliability, or any other characteristic.
 % We would appreciate acknowledgement if the document is used.
 % 
 % This document can be redistributed and/or modified freely
 % provided that any derivative works bear some notice that they are
 % derived from it, and any modified versions bear some notice that
 % they have been modified.
 % ========================================================================
 % See the file "license.terms" for information on usage and  redistribution of
 % this file, and for a DISCLAIMER OF ALL WARRANTIES.
 %  
 % ###################################################################
 %%

\documentclass[letterpaper,oldfontcommands]{memoir}
% \documentclass[]{book}
% \usepackage[]{}


\usepackage{alltt, parskip, boxedminipage} % fancyheadings, 
\usepackage{multirow, longtable, amssymb} % makeidx
\usepackage{amsmath} 

%\usepackage{multirow, longtable, tocbibind, amssymb} % makeidx, 
% \usepackage{fullpage}
%\usepackage[headings]{fullpage}
\makeindex
\usepackage[usenames]{color}
\definecolor{darkblue}{rgb}{0,0.05,0.35}
% \usepackage[dvips, pagebackref, pdftitle={FiPy}, pdfcreator={epydoc 2.1}, bookmarks=true, bookmarksopen=false, pdfpagemode=UseOutlines, colorlinks=true, linkcolor=black, anchorcolor=black, citecolor=black, filecolor=black, menucolor=black, pagecolor=black, urlcolor=darkblue]{hyperref}
\usepackage[bookmarksopen, pdftex, pagebackref, pdftitle={FiPy}, pdfcreator={epydoc 2.1}, bookmarks=true, bookmarksopen=false, pdfpagemode=UseOutlines, colorlinks=true, linkcolor=black, anchorcolor=black, citecolor=black, filecolor=black, menucolor=black, pagecolor=black, urlcolor=darkblue]{hyperref}
\usepackage{graphicx}
\usepackage{memhfixc}
% \usepackage{nameref}

\newcommand{\logo}{\rotatebox{4}{\textcolor{red}{\( \varphi \)}}\kern-.7em\raisebox{-.15em}{\textcolor{blue}{\( \pi\)}}}

\begin{document}

\settypeblocksize{9in}{7in}{*}
\setlrmargins{*}{*}{*}
\setulmargins{*}{*}{*}

% \settrimmedsize{8.5in}{11in}{*}	% pi/2
% % \settypeblocksize{40\onelineskip}{*}{0.61803}  % golden ratio
% \settypeblocksize{33\onelineskip}{*}{0.61803}  % golden ratio
% \setlength{\trimtop}{0pt}
% \setlength{\trimedge}{\stockwidth}
% \addtolength{\trimedge}{-\paperwidth}
% \addtolength{\trimedge}{-1.5in}
% \setlrmargins{*}{*}{2}
% % \setlrmargins{*}{1in}{*}
% \setulmargins{5\onelineskip}{*}{*}
% \setheadfoot{3\onelineskip}{3\onelineskip}
% \setheaderspaces{\onelineskip}{*}{*}
% \setmarginnotes{17pt}{65pt}{\onelineskip}

\checkandfixthelayout

\fixpdflayout

\setlength{\parindent}{0ex}
\setlength{\fboxrule}{2\fboxrule}
\newlength{\BCL} % base class length, for base trees.

\pagestyle{Ruled}
\renewcommand{\sectionmark}[1]{\markboth{#1}{}}
\renewcommand{\subsectionmark}[1]{\markright{#1}}

\newenvironment{Ventry}[1]%
  {\begin{list}{}{%
    \renewcommand{\makelabel}[1]{\texttt{##1:}\hfil}%
    \settowidth{\labelwidth}{\texttt{#1:}}%
    \setlength{\leftmargin}{\labelsep}%
    \addtolength{\leftmargin}{\labelwidth}}}%
  {\end{list}}

\title{
% \scalebox{10}{\logo} \\
FiPy\\
{\large A Finite Volume PDE Solver Using Python} \\
}

\author{Daniel Wheeler, Jonathan E. Guyer \& James A. Warren \\[2ex]
Metallurgy Division, Materials Science and Engineering Laboratory\\
National Institute of Standards and Technology}

\maketitle

% \include{intro}
% 
% \include{install}
% 
% \include{examples}
% 
% \include{theory}
% 
% \include{code}

\include{examples/latex/examples.diffusion.steadyState.mesh1D.input-module}

\input api

%%%%%%%%%%%%%%%%%%%%%%%%%%%%%%%%%%%%%%%%%%%%%%%%%%%%%%%%%%%%%%%%%%%%%%%%%%%
%%                                 Index                                 %%
%%%%%%%%%%%%%%%%%%%%%%%%%%%%%%%%%%%%%%%%%%%%%%%%%%%%%%%%%%%%%%%%%%%%%%%%%%%

\printindex


\end{document}
