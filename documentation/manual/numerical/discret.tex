\subsection{Discretisation}

To use the FVM, the solution domain must first be divided into
non-overlapping polyhedral Control Volumes (CV's). The values of the
independent and dependent variables located at the centres of these
CV's are assumed to be the average value across the whole CV.  The aim
of the discretisation is to reduce the continuous general equation to
a set of discrete linear equations which can then be solved to obtain
the value of the dependent variable at each CV centre.  The first step
in the discretisation of equation~\eqref{eqn:num:gen} using the FVM is
to integrate over a control volume and then make appropriate
approximations for fluxes across the boundary of each CV.  In this
section each term in equation~\eqref{eqn:num:gen} will be examined
separately. In the case of the convection and diffusion terms the
divergence theorem is used to rewrite the volume integral in terms of
area integrals over the surface of the CV.
\begin{itemize}
%
%
%
\item The transient term,
\begin{equation}
\int_V \frac{\partial (\rho \phi)}{\partial t} dV
\simeq
\frac{(\rho_{P} \phi_{P} V_P- \rho_P^O \phi_P^O V_P^O)}{\Delta t}
\label{eqn:num:tra}
\end{equation}
where the subscripts $P$ and superscripts $O$ represent the current
and previous variable value at the CV respectively.
The value, $V_P$, is the volume of
the CV and $\Delta t$ is the time step size.
%
%
%
\item The convection term,
\begin{eqnarray}
\int_V \nabla.(\rho \vec{u} \phi) dV & = & \int_S (\vec{n}.\vec{u})\rho\phi dS \\
& \simeq & \sum_{f} \rho_f (\vec{n}.\vec{u})_f \phi_f A_f
\label{eqn:num:con}
\end{eqnarray}
where $\sum_{f}$ denotes the summation over the faces of a CV and $A_f$ is
the area of each face.
The face density, $\rho_f$, is calculated using up-winding such that,
\begin{eqnarray}
\rho_f=\rho_P & \mbox{if} & (\vec{n}.\vec{u})_f \ge 0 \\
\rho_f=\rho_A & \mbox{if} & (\vec{n}.\vec{u})_f < 0 
\end{eqnarray}
where the subscript $A$ denotes the adjacent element.
The face velocity, $(\vec{n}.\vec{u})_f$ is calculated using the
Rhie-Chow~\cite{rhie}
interpolation method which will be described later in this chapter. The
vector $\vec{n}$ is the normal to the face pointing out of the CV.
The value of $\phi_f$ must depend on $\phi_A$ and $\phi_P$ when using
a first order approximation such that 
\begin{equation}
\phi_f=\alpha_f \phi_P +(1-\alpha_f)\phi_A
\end{equation}
The value $\alpha_f$ is determined by the scheme which is used~\cite{croftphd}.
First order schemes are discussed later in this chapter.
%
%
%
\item The diffusion term,
\begin{eqnarray}
\int_V \nabla.(\Gamma\nabla\phi) dV & = & \int_S \Gamma (\vec{n}.\nabla\phi) dS \\
& \simeq & \sum_f \Gamma_f (\vec{n}.\nabla\phi)_f A_f
\label{eqn:num:dif}
\end{eqnarray}
The value of the diffusion coefficient $\Gamma_f$ is estimated using
the harmonic mean~\cite{patankar}
given by,
\begin{equation}
\Gamma_f = \frac{\Gamma_A\Gamma_P}{\alpha_f \Gamma_P + (1-\alpha_f) \Gamma_A}
\end{equation}
The arithmetic mean is not used here because it over estimates the flux
when there are large differences between $\Gamma_A$ and $\Gamma_P$.
For example if either $\Gamma_P$ or $\Gamma_A$ are zero it
would be necessary for $\Gamma_f$ to be zero. If an arithmetic mean
is used this condition can not be satisfied.
The estimation for the flux, $(\vec{n}.\nabla\phi)_f$ is obtained via,
\begin{equation}
(\vec{n}.\nabla\phi)_f \simeq \frac{\phi_A-\phi_P}{d_{AP}}
\end{equation}
where the value of $d_{AP}$ is the distance between neighbouring cell centres.
This estimate relies on the orthogonality of the mesh, and becomes
increasingly inaccurate as the non-orthogonality increases. Correction terms
have been derived to improve this error, these have been described
elsewhere~\cite{croftphd}.
%
%
%
\item The source term, 
\begin{equation}
\int_V S_{\phi} dV \simeq S_\phi V_P.
\label{eqn:num:sou}
\end{equation}
If $S_\phi$ has dependence on $\phi$ then by including this dependence
stability will be increased. The dependence can only be included in
a linear manner so equation~\eqref{eqn:num:sou} becomes,
\begin{equation}
V_P (S_C - S_P \phi_P)
\end{equation}
\end{itemize}

Combining equations~\eqref{eqn:num:tra}, \eqref{eqn:num:con}, \eqref{eqn:num:dif} and
\eqref{eqn:num:sou} the complete discretisation for equation~\eqref{eqn:num:gen}
can now be written for each CV as;
\begin{eqnarray}
\frac{(\rho_{P} \phi_{P} V_P- \rho_P^O \phi_P^O V_P^O)}{\Delta t}
&+&
\sum_{f} \rho_f (\vec{n}.\vec{u})_f A_f (\alpha_f \phi_P +(1-\alpha_f)\phi_A)
\\
&=&
\sum_f \Gamma_f A_f \frac{(\phi_A-\phi_P)}{d_{AP}}
+ 
V_P ( S_C - S_P \phi_P )
\label{eqn:num:dis}
\end{eqnarray}

Equation~\eqref{eqn:num:dis} is now in the form of a set of linear combinations between
each CV value and its neighbours values.
The discretisation for the continuity equation is given by,
\begin{equation}
\frac{(\rho_P V_P-\rho_P^O V_P^O)}{\Delta t}+\sum_f F_f = 0
\label{eqn:num:cdi}
\end{equation}
Thus equation~\eqref{eqn:num:dis} using equation~\eqref{eqn:num:cdi} can be rewritten
in the following form,
\begin{equation}
a_P \phi_P = \sum_f a_{nb} \phi_{nb} + b_P
\label{eqn:num:dap}
\end{equation}
where
\begin{eqnarray}
a_P & = & V_P S_P + \frac{\rho_P^O V_P^O}{\Delta t} + \sum_f a_{nb} \\
a_{nb} & = & ( \alpha_f - 1 ) F_f + D_f \\
b_p & = & S_C + \frac{\rho_P^O V_P^O \phi_P^O}{\Delta t} 
\end{eqnarray}
The face coefficients $F_f$ and $D_f$ represent the convective strength
and diffusive conductance respectively and are given by,
\begin{eqnarray}
F_f & = & A_f \rho_f ( \vec{u}.\vec{n} )_f \\
D_f & = & \frac{A_f \Gamma_f}{d_{AP}} 
\end{eqnarray}

An area of concern when using this discretisation is non-conjunctionality
which occurs when extrapolating variables form cell centres to faces.
The value at a face is assumed to be the average value over the face.
On an unstructured mesh the face centre may not lie on the line joining
the CV centres, which will lead to an error in the face extrapolation.
To overcome this a correction has been introduced which is described
elsewhere~\cite{croftphd}. 












